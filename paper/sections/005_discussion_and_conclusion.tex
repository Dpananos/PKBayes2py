\section{Discussion}

Including age, sex, weight, and creatinine in our Bayesian model improved the inferences from that model.  Though the predicted concentrations and estimates of uncertainty changed negligibly, the covariates explained residual confounding in the random effects. This results in a model which better explains the observed variation in the data and hence should generate more plausible data for simulation.

That modes of personalization which use more information result in larger reward is ultimately unsurprising. From our perspective, the more pertinent result is the diminishing return on investment observed when using additional information (and consequently, taking on additional implementation burden to effectively use that information).  Were doses to be personalized for the simulated population, we would recommend the covariate model be used, as it is easier to implement, puts smaller burden on the clinic, and results in approximately mean/median rewards as compared to other methods.  Taking an additional sample doesn’t seem to improve the expected mean/median reward appreciably to be worth the burden of having the patient take time to come in, drawing their blood, measuring their concentration in the lab, and reporting results to the decision maker, in which a decision to not adjust the dose might be made anyway. Similar arguments can be made for Q learning, which has even higher implementation burden.

But mean/median reward does not tell the whole story.  As noted, the distribution of differences in reward is right skewed.  Some subjects have a very large difference, and the possibility of these differences might not be acceptable in different contexts with different drugs.  There is a tradeoff between less extreme differences and taking on additional clinic and implementation burden, and that tradeoff should be examined on a case by case basis.

Context is crucial, and how we adapt to that context is perhaps a question in need of closer examination.  Traditional methods of personalization include conditioning only on a subject’s covariates (not unlike the Covariate model we present here).  But of course patients are not their age, sex, weight, and creatinine.  Additionally, safety information and best available practices might change in the future as more research on drugs is performed. Were new safety information to be published, one might imagine the reward function might be affected, which may result in a new mode of personalization being more/less preferable or more/less feasible.  Any number of factors in flux can change the context in which personalization occurs, and that change in context may prompt for a re-evaluation in how personalization is done.

Thus, our results are not about apixaban per se.  We don’t offer recommendations on how personalization for apixaban should be done because we can’t anticipate the context.  What we offer is a framework for developing strategies of personalization and evaluating their performance against their implementation and clinic burden.  Context can be changed where needed, either through the reward function, or by adjusting when the clinic is able to take measurements, or by including additional information such as genotype in the Bayesian model.  Using this framework, clinics have flexibility to personalize the personalization.


\section{Limitations}

We’ve examined six modes for making decisions.  The next mode improves on a deficiency of the previous mode in a natural manner, and so our experiment constitutes a kind of ablation study.  We believe the decision making aspect of our study extracts information in a responsible way and uses the best decision making methodology available.  That being said, the experiment is not without limitations.

The bayesian model of the pharmacokinetics is integral to the methodology we present.  Any shortcomings in the model affect the quality of the decision and decision process.  Bayesian models are not as ubiquitous as other models in pharmacology, and so particular expertise is required for model development and evaluation.  That expertise increases the implementation burden of any decision process involving Bayesian models.  However, we demonstrate how one such model can be constructed in a past stdy [CITE] and include open sourced code and data for practitioners to replicate our model fitting.

Additionally, the data required to construct a high quality Bayesian model of pharmacokinetics require multiple observations of a single patient over an extended time, preferably over multiple well timed doses with near perfect adherence.  Obtaining such data requires well organized efforts and is high burden for both investigators and participating subjects.  This makes acquiring a robust Bayesian model for use in dose personalization difficult.

\section{Future Work}

Because the data required to build reliable Bayesian pharmacokinetic models are difficult to collect in practice, research into developing these models from observational data may prove fruitful in extending this work. If clinics record data on measured blood concentrations, they may have dozens or hundreds of subjects with only one or two measurements per subject.  Moreover, the subjects in question may be on multiple drugs or have comorbidities which may affect the pharmacokinetics of the drug under study.  Additional research into constructing Bayesian models which can adjust for polypharmacy and comorbidities while learning an individual’s pharmacokinetics from a large but sparse sample would drive this work towards being easier to implement in practice.


