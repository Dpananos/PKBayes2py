\section{Bayesian Models of Pharmacokinetics}

In order to estimate the optimal Q functions, we need to be able to predict how a patient's concentration is likely to evolve over time in response to a hypothetical dose (action.)  Our approach is to build a Bayesian model of patient pharmacokinetics that can use baseline clinical information, as well as any available concentration measurements, to make tailored predictions of future concentrations that are as accurate as possible given the model structure and available data. The model is flexible in that it can condition on whatever information is available - for example, if previous dose and measurement information is not available for a specific patient, the model will rely on baseline information alone. If it is available, the model will use it to (hopefully) make improved predictions. This allows us to optimize both initial doses and later dose adjustments after additional information about concentration is acquired.

We extend a previously proposed one-compartment Bayesian pharmacokinetic model \cite{pananos2020comparisons} to include fixed effects of covariates on pharmacokinetic parameters in order to incorporate baseline clinical information.  The model presented in \cite{pananos2020comparisons} is a hierarchical Bayesian model of apixaban pharmacokinetics, in which the clearance rate (L/hour), time to max concentration (hours), absorption time delay (hours), and ratio between the elimination and absorption rate constants (called alpha, a unitless parameter) are hierarchically modelled.  We extend that model by regressing the latent pharmacokinetic parameters on baseline clinical variables (age, sex, weight, and creatinine.)  To illustrate how the method works, we fit the model to previously-collected data on apixaban concentration \cite{tirona2018apixaban} and then use the fitted model to simulate patients with known "ground truth" pharmacokinetic parameters. We will then use this population of simulated patients in our experiments to explore different modes of dose personalization and their relative benefits.

The Bayesian model fit to real data, which we refer to as $ \mathcal{M}_1 $, is 

\begin{align}\label{model_M1}
	y_{i,j} &\sim \Lognormal  \left(  C_i(t_j)  , \sigma^2_y \right)  \\
	\sigma^2 &\sim \Lognormal \left( 0.1, 0.2 \right)\\	
	C_i(t_j) &= \begin{dcases}
	\frac{D_{i} \cdot F}{C l_{i}} \cdot \frac{k_{e, i} \cdot k_{a, i}}{k_{e, i}-k_{a, i}}\left(e^{-k_{a, i}\left(t_{j}-\delta_{i}\right)}-e^{-k_{e, i}\left(t_{j}-\delta_{i}\right)}\right) & t_j>\delta_i \\
	0 & \mbox{else}
	\end{dcases}\\
	k_{e,i} &= \alpha_i \cdot k_{a,i}\\
	k_{a,i} &= \dfrac{\log(\alpha_i)}{t_{max, i}\cdot(\alpha_i-1)}\\
	\delta_i &\sim \operatorname{Beta}(\phi, \kappa) \\
	\operatorname{logit}(\alpha_i) \vert \beta_\alpha, \sigma^2_\alpha &\sim \Normal(\mu_\alpha + \mathbf{x}_i^T \beta_\alpha, \sigma^2_\alpha)\\
	\log(t_{max, i}) \vert \beta_{t_{max}}, \sigma_{t_{max}} &\sim \Normal(\mu_{t_{max}} + \mathbf{x}^T_i \beta_{t_{max}}, \sigma^2_{t_{max}}) \\
	\log(Cl_i) \vert \beta_{Cl}, \sigma_{Cl} &\sim \Normal(\mu_{Cl} + \mathbf{x}^T_i \beta_{Cl}, \sigma^2_{Cl}) \\ \nonumber \\
	p(\phi) &\sim \operatorname{Beta}(20, 20)\\
	p(\kappa) &\sim \operatorname{Beta}(20, 20)\\
	p(\mu_{Cl}) &\sim \Normal(\log(3.3), 0.15^2)\\
	p(\mu_{t_{max}}) &\sim \Normal(\log(3.3), 0.1^2)\\
	p(\mu_{\alpha}) &\sim \Normal(-0.25, 0.5^2)\\
	p(\sigma_y) &\sim \Lognormal(\log(0.1), 0.2^2)\\
	p(\sigma_{CL}) &\sim \Gmma(15, 100)\\
	p(\sigma_{t_{max}}) &\sim \Gmma(5, 100)\\
	p(\sigma_{\alpha}) &\sim \Gmma(10, 100)\\
	p(\beta_{Cl, k}) &\sim \Normal(0, 0.25^2) \quad k = 1 ...	 4\\
	p(\beta_{t_{max}, k}) &\sim \Normal(0, 0.25^2) \quad k = 1 ... 4\\	
	p(\beta_{\alpha, k}) &\sim \Normal(0, 0.25^2) \quad k = 1 ... 4
\end{align}

Here, normal distributions are parameterized by their mean and variance, lognormal distributions are parameterized by the mean and variance of the random variable on the log scale, and gamma distributions are parameterized by their shape and rate.  The $\mu$ in the model above represent population means on either the log or logit scale, the $\beta$ are regression coefficients for the indicated pharmacokinetic parameter, the sigmas are the population level standard deviations on the log or logit scale, $\delta$ is aparameter which relaxes the assumption that the dose is absorbed into the blood immeditately upon ingestion, $F$ is the bioavailability of apixiban (which we fix to 0.5 \cite{byon2019apixaban}) and $D$ si the size of the dose in milligrams.  All continuous variables were standardized using the sample mean and standard deviation prior to being passed to the model.  

Once fit, $ \mathcal{M}_1$ can be used to predict the pharmacokinetics of new patients, using the patient’s covariates as predictors.  To do so, the marginal posterior distributions for $ \mu_{Cl} $, $ \mu_{t_{max}} $, $ \mu_{\alpha}$, $ \beta_{Cl} $, $ \beta_{t_{max}} $, $ \beta_{\alpha} $, $ \sigma_{Cl} $, $ \sigma_{t_{max}} $, $ \sigma_{\alpha} $, and $ \sigma_y $ must be summarized.  We use maximum likelihood on the posterior samples to summarize the marginal posterior distributions. We model the population means  and regression coefficients as normal, and the standard deviations  as gamma.  The maximum likelihood estimates are used to construct priors for a new model, which we call $ \mathcal{M}_2 $. We construct $ \mathcal{M}_2 $ so as to be able to predict plasma concentration after multiple doses (of potentially different sizes) administered over time, and remove the time delay ($ \delta $) to simplify our simulations.  Model priors for $ \mathcal{M}_2 $ are then 

\begin{align}
	p(\mu_{Cl}) & \sim \Normal(0.5, 0.04) \\
	p(\mu_{t_{max}}) & \sim \Normal(0.93, 0.05) \\
	p(\mu_\alpha) &\sim \Normal(-1.35, 0.13)\\
									\nonumber \\
	p(\sigma_{Cl}) &\sim \Gmma(69.15, 338.31)\\
	p(\sigma_{t_{max}}) &\sim \Gmma(74.96, 349.56)\\
	p(\sigma_{\alpha}) &\sim \Gmma(10.1, 102.07)\\
									\nonumber\\
	p(\beta_{Cl, 1}) &\sim \Normal(0.39, 0.08^2)\\
	p(\beta_{Cl, 2}) &\sim \Normal(0.19,0.04^2)\\
	p(\beta_{Cl, 3}) &\sim \Normal(0.02,0.04^2)\\
	p(\beta_{Cl, 4}) &\sim \Normal(0.01,0.04^2)\\
									\nonumber\\
	p(\beta_{t_{max}, 1}) &\sim \Normal(-0.01, 0.08^2)\\
	p(\beta_{t_{max}, 2}) &\sim \Normal(0.09,0.05^2)\\
	p(\beta_{t_{max}, 3}) &\sim \Normal(-0.05,0.04^2)\\
	p(\beta_{t_{max}, 4}) &\sim \Normal(-0.01,0.04^2)\\
										\nonumber\\
	p(\beta_{\alpha, 1}) &\sim \Normal(-0.19, 0.17^2)\\
	p(\beta_{\alpha, 2}) &\sim \Normal(0.33,0.11^2)\\
	p(\beta_{\alpha, 3}) &\sim \Normal(-0.06,0.1^2)\\
	p(\beta_{\alpha, 4}) &\sim \Normal(-0.09,0.1^2)\\
\end{align}

For our experiments, we generate the pharmacokinetic parameters of 1000 simulated patients from the prior predictive model of $ \mathcal{M}_2 $. Bayesian models are generative models, meaning they can generate pseudodata by drawing random variables according to the model specification going from top (model priors) to bottom (model likelihood).  To do so, we begin by resampling 1000 tuples of age, sex, weight, and creatinine from the dataset used to fit $ \mathcal{M_1} $. We sample one draw of r $ \mu_{Cl} $, $ \mu_{t_{max}} $, $ \mu_{\alpha}$, $ \beta_{Cl} $, $ \beta_{t_{max}} $, and $ \beta_{\alpha} $  from their respective prior distributions in  $ \mathcal{M}_2 $. The values of these parameters remained fixed for all 1000 patients. Conditioned on the values of these mus and betas, we compute the expectation of the population distribution for each pharmacokinetic parameter by computing $ \mu_{Cl} + \mathbf{x}^T \beta_{Cl} $, $ \mu_{t_{\max}} + \mathbf{x}^T \beta_{t_{max}} $,  $ \mu_{\alpha} + \mathbf{x}^T \beta_{\alpha} $, where $\mathbf{x}^T$ is the resampled tuple.  From the prior distribution of M2, we sample one draw of$ \sigma_{Cl} $, $ \sigma_{t_{max}} $, $ \sigma_{\alpha} $, and $ \sigma_y $.  These remained fixed for all 1000 patients. Using the previously computed expectations and $\sigma$, we sample 1000 tuples of pharmacokinetic parameters, one for each of the simulated patients.  The clearance rate and time to max concentration were sampled assuming a lognormal distribution.  Alpha was sampled using a logitnormal distribution. The pharmacokinetics can then be determined conditional on the pharmacokinetic parameters. Each of simulated patients' pharmacokinetic parameters remained fixed through the experiments.  We simulate the latent concentration using $ C(t) $ as written in $\mathcal{M}_2$, and can simulate observed concentrations by drawing a sample from a lognormal distribution with mean $\ln(C(t))$ and standard deviation $ \sigma_y$

We use Stan, an open source probabilistic programming language, for fitting our Bayesian models via Hamiltonian Monte Carlo (a Markov Chain Monte Carlo technique) and computing markov chain diagnostics. Twelve chains are initialized and run for 2000 iterations each (1000 for warmup allowing the Markov chain the opportunity to find the correct target distribution and 1000 to use as samples from the posterior).
