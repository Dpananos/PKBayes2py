\section{Introduction}

Personalized medicine has four goals: 1) to identify drugs for which between-subject variability in effectiveness or toxicity is a key issue for effective treatment, 2) to identify predictors which may explain this variability, 3) to decide on the right dose of the right drug by considering these factors, and 4) prevent adverse reactions to drugs \cite{morse2015personalized}.  Progress in all four goals has accelerated within the last decade. As an example, recent studies on DPYD genotype testing prior to starting fluoropyrimidine-based chemotherapy showed promise in preventing adverse events, making good arguments for integration of DPYD genotype testing into standard of care practices \cite{wigle2019prospective}. Despite this progress, personalized medicine still faces several barriers to widespread adoption, including economic burden, patient burden, and expertise burden required for new methods of personalization.


Personalized medicine reduces costs to the healthcare system by identifying patients who are at greater risk for adverse events or dose adjustments, thereby optimizing safety.  If the patient does not undergo an adverse event, then this ultimately saves the healthcare system the cost of the hospital stay \cite{looff2016economic}.  More ambitiously, personalized medicine has the potential to save the healthcare system costs by more effectively using resources \cite{shabaruddin2015economic}. The cost of instruments, technicians, and leadership required to operate a personalized medicine clinic are high burden, and it is not yet clear if personalized medicine is sufficiently cost effective to offset operating costs in all circumstances \cite{kasztura2019cost}.  In their 2019 scoping review of personalized medicine cost effectiveness, Kasztura et. al \cite{kasztura2019cost} found that willingness-to-pay thresholds vary wildy from country to country (citing that cost per quality adjusted life year for some modes of personalized medicine range from \$20, 000 USD per quality adjusted life year in for studies in Europe and the United Kingdom to \$200,000 USD per quality adjusted life year for studies in the United States).  This high variability in cost effectiveness means the burden required for start up may result in a positive return on investment in some areas but not others. This variability should prompt would be adopters to more closely examine if taking on the initial burden is worth the result.


% beef up here
% Examples.  Identifty clear examples from the literature around diagnostics?
The dominant perspective on personalized medicine focuses on the use of clinical and physioligical information (including biomarkers, genotyping, and diagnostic tests) as a means of optimizing treatments, but largely ignore needs, constraints, and utilities of the patient \cite{rogowski2015concepts, di2017personalized}. Patients can be burdened by frequent followup for clinical measurement (as in the case with Warfarin), be burdended by costly expenses related to obtaining care, or may be more risk adverse/tolerant than the ``typical'' patient. As an example, transportation has been found to be a large financial burden for patients recieving cancer treatment \cite{houts1984nonmedical}, and continues to burden patients, with a 2020 study finding that the cost of parking alone can climb as high as \$1600 over the course of treatment in the United States \cite{lee2020assessment}.  Additional visits to a clinic have the potential to further burden patients by requiring them to miss a day of work, and find means of childcare during their absence (if neccesary). Incorporating patient preferences and reducing the burden of personalization on the patient can result in sustained adherence \cite{elliott2008understanding}, thereby increasing effectiveness and further preventing adverse events.


% FInd some machine learning cautinary tales

An additional expertise burden is added as machine learning (used interchangeably with the term “artificial intelligence”) is adopted into personalized medicine initiatives.  Cutting edge machine learning models for prediction or decision making can be prohibitively burdensome to implement effectively. Failure to do so may result in pernicious bias inadvertently affecting subpopulations, as was found to be the case in algorithms for credit scoring \cite{barocas2016big}, crime prediction \cite{lum2016predict}, and hiring \cite{ajunwa2020paradox}.  A 2019 study found an instance of this bias in a widely used risk scoring algorithm in healthcare \cite{obermeyer2019dissecting}, demonstrating that despite the best intentions of those involved, the use of a model can lead to worse care as opposed to not using a model. Implementation of new approaches requires the partnership of experts in data science, computer science, statistics, and engineering.  Collaboration between experts in these disciplines and physicians, domain experts, and other stakeholders is crucial to make effective use of data, rigorously internally validate models, report them appropriately and transparently, and temper expectations which may be skewed from hype surrounding these algorithms \cite{frohlich2018hype}. 

These burdens may be surmountable for some, but the question then turns to if the result is worth the expense.  Answering that question is difficult without an idea how the additional burden of collecting data, or implementing new algorithms, will benefit the clinic or the patient subject to inherent constraints. 

 In this study, we present a new framework for helping practitioners interested in implementing personalized medicine to answer a) If the burden of personalization could result in more favorable outcomes for their population of patients, b) how working around patient burden (such as coming into the clinic for fewer measurements) may change effectiveness, and c) if more complex models for personalization can lead to better effectiveness. For our case study, we fit a Bayesian model to existing data on the pharmacokinetics of apixaban.  The resulting Bayesian model is used to generate synthetic pharmacokinetic data for use in experiments to compare different forms of personalization. Treating personalization as a dynamic treatment regime, we propose six policies, each increasing in complexity and clinic/patient burden, for personalizing doses of apixaban with the goal of keeping blood serum concentrations within a desired range for as long as possible. Under the assumption that the fitted Bayesian model can produce similar data to what might be observed in the future from new patients, we can make inferences as to how different policies for personalizing doses may improve upon one another, and compare if the additional burden of implementing a more complex or costly form of personalization can generate a more desirable outcome for the patient or healthcare provider.  

We begin with an overview of dynamic treatment regimes.  We then describe how to estimate an optimal dynamic treatment regime  by combining Bayesian pharmacokinetic modelling with Q-learning.  We then present our case study, beginning with the details of the Bayesian model we use to fit the real pharmacokinetic data, and present model fit diagnostics to argue that our model is satisfactory for generating synthetic data for use in our simulations. We then present and discuss the results of our simulation in light of the burden presented to a clinic to implement personalized medicine, and how this framework can be integrated to answer questions such as if personalization can produce a positive return on investment, how burdening the patient with an additional clinic visit will improve the clinic's understanding of how to treat that patient, and if using a more advanced mode of personalization will result in more favortable health outcomes.
%Given these three types of burden, want to assess if it is worth doing or not.
% Concrete examples from the three burdens 
