\section{Introduction}

% Define personalized medicine, highlight progress
Personalized medicine has been characterized by four goals: 1) to identify drugs for which between-subject variability in effectiveness or toxicity is a key issue for effective treatment, 2) to identify predictors which may explain this variability, 3) to decide on the right dose of the right drug by considering these factors, and 4) to prevent adverse reactions to drugs \cite{morse2015personalized}.  Progress in all four goals has accelerated within the last decade: For example, recent studies on DPYD genotype testing prior to starting fluoropyrimidine-based chemotherapy showed promise in preventing adverse events, making good arguments for integration of DPYD genotype testing into standard of care practices \cite{wigle2019prospective}.  

% Describe static and dynamic personalization.
% Might be nice to have a figure/schematic to show what we mean by static and dynamic personalization. Maybe check SMDM for figure limits.
With regard to the third goal---personalized dosing---the intent of most efforts has been what we call \textit{static} personalization. Such approaches inform dose at one point in time (usually induction) with the goal of eliminating ``trial-and-error'' (titration) where the dose is adapted to the patient over time in response to its effects, both therapeutic and adverse \cite{morse2015personalized}. Although significant progress has been made, for example in warfarin dosing, the need for titration has been reduced but not eliminated. Thus, there is an opportunity to personalize not only the initial doses but also the titration process to achieve the best result---we call this \textit{dynamic} personalization. This approach uses methods from disciplines such as control theory, operations research, machine learning, and biostatistics to define and apply models for optimal sequential decision-making that can select and adjust doses in response to observed impacts on the patient \textbf{cite RL healthcare examples}.

% Idea: dynamic personalization may be good, but definitely imposes burden. Our framework.
Despite its potential to improve care, dynamic personalization imposes additional burden on the patient and provider, because it requires ongoing monitoring, for example by gathering lab results, and ongoing dose adaptation. It is therefore natural to ask whether dynamic personalization is ``worth it.'' Is the additional control over dose worth the additional burden? To help answer this question, we present a unified framework for static and dynamic personalization, with two main goals.  First, to provide a framework for static and dynamic personalization based on PK modelling that can be used to create deployable models that produce clinically relevant decision support. Second, to support the evaluation of the potential benefits of static and dynamic personalization, which in turn can support decisions surrounding the implementation of a personalized medicine program. The knowledge created by our framework can be integrated into a system-level decision-making framework like Know4Go, for example, which can be used to evaluate whether such a personalized medicine program should be implemented into a particular health care system. \textit{Martin J., Lal A., Moodie J., Zhu F., Cheng D. (2016) Hospital-Based HTA and Know4Go at MEDICI in London, Ontario, Canada. In: Sampietro-Colom L., Martin J. (eds) Hospital-Based Health Technology Assessment. Adis, Cham. https://doi.org/10.1007/978-3-319-39205-9\_12}

% Case study overview.
As a case study, we use the static and dynamic personalization of apixaban dosing. We fit a Bayesian model to existing data on the pharmacokinetics of apixaban by extending previous work \cite{pananosPrevious}.  The resulting Bayesian model relates patient covariates and dose to drug concentration as a function of time. The model is then used to generate synthetic pharmacokinetic data for use in experiments to compare different forms of personalization. Implementing static and dynamic personalization as a dynamic treatment regime \citep{chakraborty2013statistical}, we propose six policies, each increasing in complexity and clinic/patient burden, for personalizing doses of apixaban with the goal of keeping blood serum concentrations within a desired range for as long as possible. Under the assumption that the fitted Bayesian model can produce similar data to what might be observed in the future from new patients, we can assess the benefits of the different policies, and hence assess how those benefits relate to their associated burdens.

%Paper structure
The paper proceeds as follows. We begin in Section~\label{ss:dtrs} with an overview of dynamic treatment regimes, which underpin our models for dynamic personalization.  We then describe in Section~\label{ss:optimal} how to estimate an optimal dynamic treatment regime by combining Bayesian pharmacokinetic modelling with Q-learning. In Section~\label{ss:framework}, we present our framework for assessing different levels of static and dynamic personalization.  We then present our case study in Section~\label{ss:casestudy}, beginning with the details of the Bayesian model we use to fit the real pharmacokinetic data, and present model fit diagnostics to establish that our model is satisfactory for generating synthetic data for use in our simulations. Finally in Section~\label{ss:} we present and discuss the results of our simulation in terms of the benefit/burden tradeoff.
