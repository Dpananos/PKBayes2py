\section{Introduction}

Personalized medicine has four goals: 1) to identify drugs for which between-subject variability in effectiveness or toxicity is a key issue for effective treatment, 2) to identify predictors which may explain this variability, 3) to decide on the right dose of the right drug by considering aforementioned factors, 4) and to aid in the prevention of adverse reactions of said drugs \cite{morse2015personalized}.  Progress in all four goals has accelerated within the last decade. As an example, recent studies on DPYD genotype testing prior to starting fluoropyrimidine-based chemotherapy showed promise in preventing adverse events, making good arguments for integration of DPYD genotype testing into standard of care practices \cite{wigle2019prospective}. Despite this progress, personalized medicine still faces barriers to widespread adoption.

The cost of instruments, technicians, and leadership required to operate a personalized medicine clinic are non-negligible, and it is not yet clear if personalized medicine is sufficiently cost effective to offset operating costs \cite{kasztura2019cost}. Additionally, some perspectives on personalized medicine focus on the use of demographic and clinical/biological information (including biomarkers, genotyping, and diagnostic tests) as a means of optimizing treatments, but largely ignore needs, constraints, and utilities of the patient \cite{di2017personalized} (for example their availability or willingness to be subject to multiple blood draws to measure blood serum concentrations). Ignoring these constraints may result in a method of personalization which, although effective, may not be realistic to implement at scale because the cost to the clinic, or the burden to the patient, is just too large.

An additional expertise cost is added as machine learning (used interchangeably with the term “artificial intelligence”) is adopted into personalized medicine initiatives.  Cutting edge machine learning models for prediction or decision making can be prohibitively complex to implement correctly and at scale, requiring the partnership of experts in data science, computer science, statistics, engineering, etc.  Collaboration between experts in these disciplines and physicians (among other stakeholders) is crucial to make effective use of data, rigorously internally validate models, and temper expectations which may be skewed from hype surrounding these algorithms \cite{frohlich2018hype}.  Thus, new approaches to personalization may be out of reach without financial means to hire experts should they not be available for collaborative work (vis \`a vis large research grants, etc).  

These costs may be payable for some, but the question then turns to if the result is worth the expense.  Answering that question is difficult without an idea how the additional cost of collecting data, or implementing new algorithms, will benefit the clinic or the patient subject to inherent constraints.  In this study, we present a new framework for helping practitioners interested in implementing personalized medicine to answer these questions while considering practical limitations, and we present a case study using the framework applied to apixaban dosing.

For our case study, we fit a Bayesian model to existing data on the pharmacokinetics of apixaban.  The resulting Bayesian model is used to generate synthetic pharmacokinetic data for use in experiments to compare different forms of personalization. Treating personalization as a dynamic treatment regime, we propose six policies, each increasing in complexity and clinic/patient burden, for personalizing doses of apixaban with the goal of keeping blood serum concentrations within a desired range for as long as possible. Under the assumption that the fitted Bayesian model can produce similar data to what might be observed in the future from new patients, we can make inferences as to how different policies for personalizing doses may improve upon one another, and compare if the additional burden of implementing a more complex or costly form of personalization can generate a more desirable outcome for the patient or healthcare provider.  

We begin with an overview of dynamic treatment regimes and how personalizing doses can be framed as such.  Afterwards, we describe how estimating an optimal policy for a dynamic treatment regime can be done using Q learning (the most complex method we entertain here).  We discuss the details of the Bayesian model we use to fit the real pharmacokinetic data, and present model fit diagnostics to argue that our model is satisfactory for generating synthetic data for use in our simulations. We then present and discuss the results of our simulation in light of the costs presented to a clinic to implement personalized medicine, and how this framework can be integrated to make decisions about what form of personalization to implement.


