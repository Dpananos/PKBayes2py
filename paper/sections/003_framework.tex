\section{A Framework for Assessing Static and Dynamic Personalization}\label{ss:framework}

[Words about summary of framework. Should we describe it as "steps" below? Numbered?]

\begin{itemize}
\item Bayesian Modelling
\begin{itemize}
	\item Fit model
	\item Assess model quality
	\item Simulate data
\end{itemize}
\item Define Possible Modes of Personalization
\begin{itemize}
	\item No personalization
	\item Titration only - no covariates
	\item Static - covariates only
	\item Simple dynamic
	\item Complex dynamic
\end{itemize}
\item Evaluate Benefits using Simulated data
\begin{itemize}
\item Steps?
\end{itemize}
\end{itemize}

\subsection{Bayesian Modelling}

The first step in our framework is to fit a Bayesian model that relates patient covariates and dose to drug concentration as a function of time. For example, previous work \cite{pananos2020comparisons} describes a hierarchical Bayesian model of apixaban pharmacokinetics, in which the clearance rate (L/hour), time to max concentration (hours), absorption time delay (hours), and ratio between the elimination and absorption rate constants (called alpha, a unitless parameter) are hierarchically modelled.  That model by regressing the latent pharmacokinetic parameters on baseline clinical variables (age, sex, weight, and creatinine) to permit personalization.

Having fit the model, it is important to assess the quality of the result; several diagnostics are commonly used including...\textbf{Go on about this for a little while.}

The model is then used to generate synthetic pharmacokinetic data for use in experiments to compare different forms of personalization. Each generated data point may be thought of as one synthetic patient, with observed covariates and observed pharmacokinetic parameters. These parameters, which are never observed in real data, allow us to compute the effects of any dosing decisions (which are made \textit{without} direct knowledge of the parameters), and thus allow us to evaluate the performance of different modes of personalized dosing on the sampled population. 

\subsection{Modes of Personalization}

The second step in our framework is to identify modes of personalization that we wish to evaluate. This would typically include both static and dynamic modes of personalization. For example, \textbf{go on about this for a while.}

\subsection{Evaluation}

\textbf{Describe training/testing here I think.}

We compare all methods on their achieved reward as well as the difference between the achieved reward and theoretically largest reward the reward we would achieve if we knew the pharmacokinetic parameters exactly).  Because we know the true latent pharmacokinetic parameters of the simulated subjects, we can optimize the reward with the known pharmacokinetics of the subject, thereby yielding the largest reward possible.